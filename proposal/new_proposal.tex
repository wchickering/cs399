\documentclass[10pt]{article}

% =========================================================================
% document style changes
% =========================================================================

\usepackage{amsmath}                    % AMS math packages
\usepackage{amssymb}                    %
\usepackage[]{graphpap}
\usepackage[T1]{fontenc}                % for \mathrm{}
\usepackage{courier}                    % for \texttt{}
\usepackage{bbm}                        % for \mathbbm{1} (indicator function)
\usepackage{booktabs}
\usepackage{graphicx}
\usepackage{caption}
\usepackage{subcaption}

%\setlength{\parskip}{\baselineskip}     % skip line following paragraphs
\setlength{\parskip}{0.0in}
\setlength{\parindent}{0.3in}           % Control margins and amount of text
\setlength{\topmargin}{-0.8in}
\setlength{\oddsidemargin}{-.15in}      % changed from {-.15in}
\setlength{\textheight}{9.5in}
\setlength{\textwidth}{6.8in}
\pagestyle{empty}                       % No page numbers

\newcommand{\spc}{\vspace{0.25in}}      % Shortcut commands
\newcommand{\ds}{\displaystyle}         %\newcommand{\ds}[1]{\displaystyle{#1}}
\newcommand{\ra}{\rightarrow}

\begin{document}                        % This is where the document begins

{\LARGE\bf
\begin{tabbing}
\hspace{2.8in} \= \hspace{1.3in} \= \hspace{1.2in} \= \\

%=========    Heading   ==================================================
CS 399 \> PA4 \> Bill Chickering (bchick)\\
\normalsize Jan 18, 2014 \> \> Jamie Irvine (jirvine)
% =========================================================================
\end{tabbing}
} \vspace{.4in}

\section*{Abstract}
\emph{Will write this when the paper is done}

\section*{Introduction}
A key challenge in retail is choosing which items to incorporate into one's
collection of offered goods and services. These decisions can significantly
impact a retailer's revenue in the short run as well as effect customer
experience, and hence, their relationship with the retailer. Several factors
must be considered including the number of product lines, the vareity of
products in each line, as well as the consistency and relationships between
products. In this study, we focus on the latter and explore techniques that
leverage publicly available product recommendation information for the purpose
of improving product assortment decisions.

It has become standard practice for online retailers to recommend one or more
products to prospective customers who have viewed or purchased an item. These
recommendations are generally derived from collaborative filtering or other
data-mining techniques employed by the retailer or a third-party service
provider. Importantly, these recommendations provide a source of publicly
available business intelligence by relating the items in their catalogue.
Specifically, these online recommendations logically form a directed graph in
with the vertices are products and an edge pointing from item A to item B
indicates that customers who view or purchase item A are recommended item B.
Given the recommendation graphs of two or more distinct retailers, our goal is
to determine new, meaningful edges that connect these graphs in a way that would
allow one retailer to relate their products to those of another retailer.

In this study we choose to focus on department stores, which typically have
overlapping product catalogues. We intend to exploit this overlap by identifying
identical inter-retailer products in order to achieve some initial connectivity
between their recommendation graphs. From there, we will explore and compare a
vareity of techniques for choosing additional edges to further connect these
originally disconnected graphs. In addition to relating {\em inter-retailer} items, we
are also interested in leveraging other retailer's recommendation graphs in
order to determine new {\em intra-retailer} edges, which can be valuable for
improving customer recommendations.

\section*{Data}
We collect the data by scraping the websites of major departments stores for
products with their descriptions, prices, photos, etc. From each product page,
we also gather the website's top recommended items for that product.  For this
paper, we scraped the catalogues of Bloomingdale's, Neiman Marcus, Nordstrom,
etc.. Each of these websites requires their own scraper, but the same data is
accessible from all of them, including at least four recommended products for
each product in the catalogue. This allows us to build a directed recommendation
graph for each department store. 

It is worth noting that in reality, a company would have more data about their
own products than what they post on their website. However, the recommendation
graph is still quite informative and is accessible to anyone.  It is realistic
that a company could gather graphs like this from each of their competitors, as
we have here.

\section*{Methods}
We compare three approaches to extending the recommendation graphs. The first
approach is a baseline that uses a content-driven similarity function to add
edges between items. It ignores any information from the original
recommendation graphs. The second approach focuses exclusively on the
recommendation graphs. This method uses an identity function to determine which
items in different graphs are the same and thus links the graphs together.
Using a random walk algorithm, we determine which unlinked items should be
linked. The third approach is uses a combination of the first two to leverage
both kinds of information.

\subsection*{Method 1}
Create similarity scores between items. If items are similar, we add edges in
both directions to the recommendation graph. To determine the threshold of how
high the score must be before adding an edge, we calculate the average
similarity between items that are connected in the original graph.  These edges
can connect two items in the same catalogue or items across catalogues.

\emph{More about the specifics of the function here}

\subsection*{Method 2}
The second method uses the recommendation graphs themselves to determine new
recommendation edges. Since there is a significant overlap of products between
department stores (x\% between Macy's and Bloomingdales for example),
recognizing which products are the same connects the otherwise disjoint graphs
of different companies. To identify which products are identical, we construct
an identity function. \emph{This function works like this and was tested like
this, so we know it works.}

With connected graphs, we can apply a random walk algorithm to determine which
new edges would be most appropriate to add. In the random walk, each product
node is a source of a 'walker', which has equal probability of moving down each
outgoing edge. We add edges based on the probability that a walker goes from
one product to another. Since the bench-mark for an appropriate threshold is
less obvious in this method, we use the top X scores, the number of edges above
the threshold found in method 1. While this is somewhat arbitrary, it allows
for a fair comparison between the methods.

\emph{More about the specifics of the random walk algorithm here}

\subsection*{Method 3}
The final method leverages both content-similarity and the structure of the
recommendation graphs. First, we combine the recommendation graphs using an
identity function and run a random-walk algorithm, as in method 2. This
determines how 'far,' in a sense, any product is from any other product. To
determine if there should be an edge added from product A to product B, we use
the content-similarity function of method 1 to compare A to every other product
in the catalogue and sum these scores up with the weights from the random-walk
originating at product B. Thus content-similarity from product A to a product
that is near product B in the recommendation graph increases the likelihood
that product A should be connected to product B. \emph{An equation would
probably be good in here}. For this method, we also add the top X edges based
on the threshold in method 1.

\emph{More about the specifics of the implementation here}

\section*{Evaluation}
Empirically evaluating the methods is somewhat difficult with the data we have
access to. With all of the user-traffic data, one could see if an added
recommendation edge connects two products that are in fact co-clicked on
co-purchased. Without this, the question of whether one product should be
recommended for another can be answered with more subjective human judgement.

For this experiment, we collect the top edges to be added for each of the three
methods. We show a subset of them to a number of people who rate them in
exchange for small sexual favors. The tester is asked 'How true is the
following statement: If someone likes product 1, they will probably like
product 2', followed by the pictures and descriptions of the two items of the
directed edge. The tester answers on a 5 point scale and the results are
aggregated. Each tester is shown edges from each of three methods, to decrease
bias from a biased tester.

\emph{More about how many testers we use}

\section*{Results}
\emph{We'll see}

\section*{Conclusion}
\emph{Will write this when the paper is done}

\end{document}
