\documentclass[10pt]{article}

% =========================================================================
% document style changes
% =========================================================================

\usepackage{amsmath}                    % AMS math packages
\usepackage{amssymb}                    %
\usepackage[]{graphpap}
\usepackage[T1]{fontenc}                % for \mathrm{}
\usepackage{courier}                    % for \texttt{}
\usepackage{bbm}                        % for \mathbbm{1} (indicator function)
\usepackage{booktabs}
\usepackage{graphicx}
\usepackage{caption}
\usepackage{subcaption}

%\setlength{\parskip}{\baselineskip}     % skip line following paragraphs
\setlength{\parskip}{0.0in}
\setlength{\parindent}{0.3in}             % Control margins and amount of text
\setlength{\topmargin}{-0.8in}
\setlength{\oddsidemargin}{-.15in}       % changed from {-.15in}
\setlength{\textheight}{9.5in}
\setlength{\textwidth}{6.8in}
\pagestyle{empty}                       % No page numbers

\newcommand{\spc}{\vspace{0.25in}}      % Shortcut commands
\newcommand{\ds}{\displaystyle}         %\newcommand{\ds}[1]{\displaystyle{#1}}
\newcommand{\ra}{\rightarrow}

\begin{document}                        % This is where the document begins

{\LARGE\bf
\begin{tabbing}
\hspace{2.8in} \= \hspace{1.3in} \= \hspace{1.2in} \= \\

%=========    Heading   ==================================================
CS 399 \> PA4 \> Bill Chickering (bchick)\\
\normalsize Jan 18, 2014 \> \> Jamie Irvine (jirvine)
% =========================================================================
\end{tabbing}
}
\vspace{.4in}

The abstract goes here. . . . 
Blah blah blah.  Blah blah blah.  Blah blah blah.  Blah blah blah.  Blah blah
blah.  Blah blah blah.  Blah blah blah.  Blah blah blah.  Blah blah blah.  Blah
blah blah.  Blah blah blah.  Blah blah blah.  Blah blah Blah blah blah.  Blah
blah blah.  Blah blah blah.  Blah blah blah.  Blah blah Blah blah blah.  Blah
blah blah.  Blah blah blah.  Blah blah blah.  Blah blah Blah blah blah.  Blah
blah blah.  Blah blah blah.  Blah blah blah.  Blah blah Blah blah blah.  Blah
blah blah.  Blah blah blah.  Blah blah blah.  Blah blah Blah blah blah.  Blah
blah blah.  Blah blah blah.  Blah blah blah.  Blah blah blah.  Blah blah blah.
Blah blah blah.  Blah blah blah.  blah.  Blah blah blah.  Blah blah blah.  Blah
blah blah.  blah.  Blah blah blah.  Blah blah blah.  Blah blah blah.  blah.
Blah blah blah.  Blah blah blah.  Blah blah blah.  blah.  Blah blah blah.  Blah
blah blah.  Blah blah blah.  blah.  Blah blah blah.  Blah blah blah.  Blah blah
blah.

\section*{Vision}

Shopping for lifestyle items like apparel, shoes, glasses, watches, etc. calls
for a distinct user experience. The shopper often has a fuzzy idea of what they
want--Emma wants a black party dress or Paul wants printed T-shirts--so they'd
like to peruse many similar items before deciding what to purchase. Today, it's
difficult to achieve this without having 1) a huge product catalog 2) an
excellent search engine 3) an exceptionally well articulated query. Certainly
(1) exists. Arguably (2) exists. But (3) is an onerous proposition. Instead,
one typically wades through a large number of heterogenous items hoping to find
something they like.

We're debating the details of the user interface, but the shopping app we want
on our phones would resemble the following. You begin by either entering a
query or clicking on a highlighted item. You are then shown an item. If you
like the item and want to see more like it you swipe it to the right. If you
don't like it, you swipe it to the left. If you really like it, you swipe it
down into your cart/list. You can swipe through items quickly--less than a
second per item. As you shop, the system learns in real time about your likes
and dislikes, showing you items you like with increasing frequency. At any
time, you can toggle between your search, your cart/list, or even the lists of
everything you liked or disliked. Click any item anywhere, and you see product
details along with an option to buy. Lists of items can be posted on Pinterest,
Tumblr, Facebook, or emailed to yourself or a friend. Or save a list for later
and continue a shopping session at any time.

We believe a critical ingredient to maximal user-engagement is rapid feedback
and responsiveness. People need less than a second or two to decide whether
they're interested in a pair shoes. And every time a user indicates interest or
disinterest in an item the system learns about the user's immediate desires and
is therefore better equipped to identify desirable items. Meanwhile, the system
can leverage user-sessions in aggregate to learn about item popularity and
item-item associations.  We believe this last idea is key. Learning item-item
associations from user activity is what our CS 341 work was all about. In that
project, we combined aggregated user activity along with a particular user's
click history to improve search results. The app described here would work
similarly. Importantly, however, this app idea attempts to address a data
sparsity issue by creating a user interface that by its nature will collect far
more user activity than a traditional search engine.

\section*{Implementation Ideas}

\end{document}
